\documentclass[10pt, a4paper, fleqn]{article}


\usepackage[utf8]{inputenc} % utf8x kann alle
                                % Textcodierungen
                                % interpretieren
\usepackage[T1]{fontenc} % Schriftcodierung mit UTF-8
\usepackage{textcomp} % Erweiterung von fontenc

\usepackage[ngerman]{babel}

\usepackage{tikz}
\usepackage{circuitikz}

\begin{document}

\section{Kapazitätsmessung einer Diode}

Wir nehmen ein LC-Meter und messen damit die Kapazität einer Diode.
Hierzu entkoppeln wir die Schaltung mit einer Kapazität gegenüber dem
LC-Meter, so werden die DC-Anteile hin zum LC-Meter gefiltert. 
Auf der anderen Seite schliessen wir eine DC-Spannungsquelle
$V_q$ an um die Diode in Sperrrichtung zu betrieben. Die Spannung 
der Quelle liegt dabei im Bereich $0-40V$, was für eine Diode vom Typ
1N4007 weit unter deren Durchbruchspannung liegt, d.h. wir keinen 
Kurzschluss damit erzeugen. Der Wirderstand zwischen Quelle und 
Diode dient der Entkopplung (dies war zuerst mit einer Induktivität
realisiert worden, hatte aber die Messung negativ beeinflusst).

\begin{figure}[h!]
	\centering
	\begin{circuitikz}[scale=1]\draw
		% (0,0) node[anchor=east] {GND}
		% (0,0) to[short, o-*] (2,0)
		(2,0) to[C=$C_L$, *-*] (2,2)
		(1,2) node[anchor=east] {LC-Meter}
		(1,2) to[short, o-*] (2,2)
		(2,2) to[C=$100nF$, *-*] (4,2)
		(2,0) to[short, *-*] (4,0)
		(4,0) to[D, l_=1N4007, *-*] (4,2)
		(4,2) to[R=$100k\Omega$, *-] (8,2)
		(4,0) to[short, *-] (8,0)
		(8,0) to[V, l_=$V_q$, ] (8,2)
		(2,0) node[ground] {};
	\end{circuitikz}
	\caption{Messschaltung}
\end{figure}

Mit diesem Aufbau wurde dann die Kapazizät ermittelt zu verschiedenen
Spannungen von $V_q$. 



\end{document}
