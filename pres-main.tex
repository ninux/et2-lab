\documentclass{beamer}

\usepackage[utf8]{inputenc}
\usepackage[ngerman]{babel}
\usepackage{beamerthemeshadow}
\usepackage{calc}
\usepackage{ifthen}
\usepackage{tikz}
\usepackage{caption}
\usepackage{subcaption}
\usepackage{bchart}
\usepackage{wrapfig}
\usepackage{ulem}
\usepackage{color}

\usepackage[
	siunitx,
	european,
	europeanvoltages,
	europeanresistors,
	europeanports
	]{circuitikz}


\definecolor{darkgreen}{rgb}{0.2,0.6,0.2}

\usepackage{listings}

\title{Laborübung Parallelschwingkreis}
\subtitle{ET2-Labor, Teil 2}
\author{E. Mazlagi\'c, A. Schmid}
\institute{Hochschule Luzern \\ Technik \& Architektur}

\begin{document}
\maketitle

% Listing settings
\lstset{
    breakwhitespace=true,
    language=[LaTeX]TeX,
    basicstyle=\footnotesize\ttfamily,
    keywordstyle=\color{red}\bfseries,
    idnetifierstyle=\color{blue},
    commentstyle=\color{darkgreen},
    stringstyle=\color{blue},
    columns=fullflexible,
    keepspaces=true,
    breaklines=true,
    tabsize=3,
    showstringspaces=false,
    extendedchars=truei
    morekeywords={
	\begin, 
	\item,
	\end}
}

\section{Aufgabenstellung}

\subsection{Schaltung}

\begin{frame}
	\frametitle{Schaltung}
	\begin{figure}
		\centering
		\begin{circuitikz}[scale=1.1]\draw
			(0,0) node[anchor=north]{B}
			to[short, o-*](2,0)
			to[L, l_=10mH, *-*](2,2)
			to[R, l_=100k$\Omega$, *-o](0,2)
			(2,2) to[short, *-*](4,2)
			to[C=10nF, *-*](4,0)
			(2,0) to[short, *-*](4,0)
			(4,0) to[short, *-o](6,0)
			(4,2) to[short, *-o](6,2)
			(6,2) node[anchor=south]{C}
			(6,0) node[anchor=north]{D}
			(0,2) node[anchor=south]{A}
			(-2,2) to[sI, l^=$f\neq kosnt.$](-2,0)
			(-2,2) to[short](-1,2)
			(-2,0) to[short](-1,0)
			(6,0.2) to[open, v=$v(f)$] (6,1.8);
		\end{circuitikz}
		\caption{Schaltung aus Aufgabe 5, ET2-Labor Teil 2}
	\end{figure}
\end{frame}

\section{Resultate}

\subsection{Beobachtung}
\begin{frame}
	\begin{block}{Was beobachtet man bei Grenzfrequnez?}
		\begin{itemize}
			\item{$\varphi = 0$}
			\item{$\hat u_{Out} =$ maximal}
		\end{itemize}
	\end{block}
\end{frame}

\subsection{$f_0$ berechnet}
\begin{frame}
	\begin{block}{Werte beim berechneten Wert für $f_0$}
		\begin{itemize}
			\item $\varphi = 15^o$
			\item $\hat u_{Out} \neq$ maximal
		\end{itemize}
	\end{block}
\end{frame}

\subsection{$f_0$ ermittelt}
\begin{frame}
	\begin{block}{Werte beim ermittelten Wert für $f_0$}
		\begin{itemize}
			\item $f_0 = 15.78kHz$
			\item $\hat u_{Out} =$ maximal
		\end{itemize}
	\end{block}
\end{frame}


\end{document}
