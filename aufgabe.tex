\section{Aufgabenstellung}

\subsection{Schaltung}

\begin{frame}
	\frametitle{Schaltung}
	\begin{figure}
		\centering
		\begin{circuitikz}[scale=1.1]\draw
			(0,0) node[anchor=north]{B}
			to[short, o-*](2,0)
			to[L, l_=10mH, *-*](2,2)
			to[R, l_=100k$\Omega$, *-o](0,2)
			(2,2) to[short, *-*](4,2)
			to[C=10nF, *-*](4,0)
			(2,0) to[short, *-*](4,0)
			(4,0) to[short, *-o](6,0)
			(4,2) to[short, *-o](6,2)
			(6,2) node[anchor=south]{C}
			(6,0) node[anchor=north]{D}
			(0,2) node[anchor=south]{A}
			(-2,2) to[sI, l^=$f\neq kosnt.$](-2,0)
			(-2,2) to[short](-1,2)
			(-2,0) to[short](-1,0)
			(6,0.2) to[open, v=$v(f)$] (6,1.8);
		\end{circuitikz}
		\caption{Schaltung aus Aufgabe 5, ET2-Labor Teil 2}
	\end{figure}
\end{frame}
